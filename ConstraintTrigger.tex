\ifx\PREAMBLE\undefined
\input{preamble}
\begin{document}
\fi
\chapter{Constraints and Triggers}
Constraints, or integrity constraints constrain the allowable states of the database, while triggers monitor database changes, check conditions and possibly initiates actions on being activated. 
\section{Overview}
\subsection{Constraints}
\textbf{Integrity constraints} impose restrictions on allowable data beyond those imposed by the structure of the DB and types of the attributes. For example, in our \texttt{Student} table, the \texttt{GPA} value is required to fall inside (0.0, 4.0]; in the \texttt{Apply} table and the value of \texttt{decision} should be either \texttt{Y}, \texttt{N} or \texttt{Null}. Constraints can be logically more complicated, e.g. the condition that no decisions have been made for CS applicants can be expressed as the constraint \texttt{major = `CS' $\Rightarrow$ decision = Null}. 

Constrains are enforced for several purposes:
\begin{itemize}
\item To automatically catch data-entry error when doing insertions. 
\item As correctness criteria when doing updates. 
\item To enforce consistency.
\item To inform the system of properties of the data so as to facilitate better storage and queries.
\end{itemize}

Constraints are classified into:
\begin{itemize}
\item Non-null constraints. An attribute can be forbidden from having null value.
\item Key constraints. Attributes specified as key should have unique value combinations. 
\item Referential integrity (foreign key).
\item Attribute-based constraints.
\item Tuple-based constraints.
\item General assertions.
\end{itemize}

Constraints can be declared either when the schema is created or later. In the former case, it is checked after bulk loading, while in the latter case it is checked on the state of the DB at the time of the declaration. 

Normally a constraint should be checked after every modification that may cause a violation to it. Constraint checking can also be carried out only after each transaction, which is called deferred constraint checking: constraints are allowed to be violated during the transaction, but if any violation still exists after the transaction is completed, the whole transaction gets rolled back. 
\subsection{Trigger}
\textbf{Trigger} is a more dynamic concept than constraints. Constraints put restrictions on each state of the DB, while triggers limit how the DB evolves. Triggers are also called ``Event-Condition-Action Rules'' because they usually have the form 
\begin{center}
When \emph{event} occurs, check \emph{condition}; If true, do \emph{action}. 
\end{center}
For example, a trigger can be set up to accept applications from student with a GPA  higher than 3.95 automatically:
\texttt{insert application with GPA$>$3.95 $\Rightarrow$ accept}.

Triggers are enforced for the following purposes:
\begin{itemize}
\item To move logic from application into DBMS.
\item To enforce constraints. In mainstream implementations, the trigger feature is more expressive than the constraint feature. Moreover, triggers make it possible to repair the DB when a constraint violation is found.
\end{itemize}
\section{SQL for Constraints}
\subsection{Non-null Constraints}
To add a non-null constraint to an attribute when creating the table, we need to add the keyword \texttt{not null} to the attribute. 
\begin{lstlisting}
create table Student(sID int, sName text, GPA real not null, sizeHS int);
\end{lstlisting}
\subsection{Key Constraints}
A key constraint requires that values of the specified attribute be unique.
\begin{lstlisting}
create table Student(sID int primary key, sName text, GPA real, sizeHS int);
\end{lstlisting}
Suppose the table contains two tuples:
\begin{lstlisting}
234 Bob 3.6 1500
123 Amy 3.9 1000
\end{lstlisting}
The statement
\begin{lstlisting}
update Student set sID = sID - 111;
\end{lstlisting}
is possible to cause an error because Bob will have the same sID as Amy if it gets updated first. This demonstrates the situation in which deferred constraint checking should be used.

There can be only one primary key in each table. If attributes other than the primary key is supposed to be unique, the keyword \texttt{unique} should be used.
\begin{lstlisting}
create table Student(sID int primary key, sName text unique, GPA real, sizeHS int);
\end{lstlisting}
The following examples demonstrate the case in which a combination of attributes is used as the primary key.
\begin{lstlisting}
create table College(cName text, state text, enrollment int, primary key (cName,state));
create table Apply(sID int, cName text, major text, decision text, primary(sID, cName), unique(sID, major));
\end{lstlisting}
Note that Null values are handled specially. SQL standard and most implementations allow repeated Null values even if the column is declared as unique. However repeated Null values are usually not permitted for the primary key.
\subsection{Attribute/Tuple Based Constraints}
Attribute-based constraints are often used to catch data entry errors concerning ranges of attributes. They are check whenever a tuple is inserted or updated. In the following example we put limits to the range of GPA and sizeHS. 
\begin{lstlisting}
create table Student(
	sID int, sName text,
	GPA read check(GPA <= 4.0 and GPA > 0.0),
	sizeHS int check(sizeHS < 5000));
\end{lstlisting}
Tuple-Based constraints also catches data entry errors, but it puts limits to relationship of different attributes inside a tuple. In the following example, the tuple-based constraint rules out successful applications for CS major of Stanford.  
\begin{lstlisting}
create table Apply(sID int, cName text, major text, decision text,
	check(decision = 'N' or cName <> 'Stanford' or major <> 'CS'));
\end{lstlisting}
Attribute/Tuple based constraints may be syntactically allowed but not enforced in MySQL. Currently sub-queries are allowed in check constraints in no implementation, although they are listed in SQL standard and might be syntactically permitted in certain implementations.  
\subsection{General Assertion}
General assertions are very powerful, but unfortunately not supported in any mainstream DBMS. In the SQL standard it takes the form
\begin{lstlisting}
--A is unique
create assertion Key
check ((select count(distinct A) from T) =
       (select count(*) from T));
--Apply only contains sID from Student
create assertion ReferentialIntegrity
check (not exists (select * from Apply
	where sID not in (select sID from Student)));
--Average GPA of admitted students higher than 3.0
create assertion AvgAccept
check (3.0 < (select avg(GPA) from Student 
	where sID in (select sID from Apply 
		where decision = 'Y')));
\end{lstlisting}
The complexity of implementing general assertion comes from the requirement that it has to be checked each time any modification that can possibly cause violation is made to the DB.
\subsection{Referential Integrity}
Referential integrity handles the integrity of references in a database. No ``dangling pointer'' is allowed. In our example, it should be guaranteed that an \texttt{sID} in \texttt{Apply} refers to a student in \texttt{Student}, and that a \texttt{cName} in \texttt{Apply} refers to a college in \texttt{College}, i.e. we have referential integrities \texttt{Apply.sID to Student.sID} and \texttt{Apply.cName to College.cName}, as shown below. Note that referential integrity is a directional constraint. 
\begin{lstlisting}
create table Student(sID int primary key, sName text, GPA real, sizeHS int);
create table College(cName text primary key, state text, enrollment int);
create table Apply(sID int references Student(sID), 
				cName text references College(cName),
				major text, decision text);
\end{lstlisting}

In a referential integrity \texttt{R.A to S.B}, \texttt{A} is called the \textbf{foreign key}. \texttt{B} is usually required to be the primary key of \texttt{S}, or at least a \texttt{unique} attribute. Multi-attribute foreign key is allowed. 

The enforcement of referential integrity \texttt{R.A to S.B} should consider possible violations by the following operations: 
\begin{itemize}
\item insert into \texttt{R};
\item delete from \texttt{S};
\item update \texttt{R.A};
\item update \texttt{S.B}.
\end{itemize}
For insertion to \texttt{R} and update of \texttt{R.A}(the referencing table), the referential integrity is simply checked. The referential integrity also prevents us from dropping table \texttt{S}. For deletion from \texttt{S}, three types of actions are possible:
\begin{description}
\item[Restrict(default)]Generate an error.
\item[Set Null]Set referencing tuples in \texttt{R} to have Null value for \texttt{R.A}.
\item[Cascade]Delete all referencing tuples in \texttt{R}. 
\end{description}
Actions for update of \texttt{S.B}(the referenced table) are similar, except that for cascade, the same update is propagated to \texttt{R.A} in \texttt{R}. The actions can be set in the form
\begin{lstlisting}
create table Apply(sID int references Student(sID) on delete set null, 
				cName text references College(cName) on update cascade,
				major text, decision text);
\end{lstlisting}
We can have referential integrities within the same table:
\begin{lstlisting}
create table T(A int, B int, C int, D int, primary key(A,B), 
	foreign key (B,C) references T(A,B) on delete cascade);
\end{lstlisting}
\section{SQL for Triggers}
Implementations of trigger vary significantly across different systems. We will talk about triggers in the SQL standard. The syntax of trigger is as follow.
\begin{lstlisting}
Create Trigger name 
Before | After | Instead of events
[referencing-variables]
[For Each Row]
when (condition)
action
\end{lstlisting}
The trigger should be activated \texttt{before/after/instead of} the \texttt{events} listed in the events clause, which can be insert on table T, delete on table T or update of a few columns(optional) on table T. Each event can involve several tuples. If \texttt{For Each Row} is specified, the trigger is activated for each of these tuples, while the trigger is activated only once if there is no \texttt{For Each Row}. The \texttt{referencing-variables} clause provides a method to access the data whose modification activated the trigger. Old rows, new rows, old tables and new tables can be referenced, with clauses of the form \texttt{old row/new row/old table/new table as var}. Note that \texttt{old/new row} can only be referenced when the event is row-level (its contrary is statement-level), i.e. when \texttt{For Each Row} is enabled. \texttt{condition} and \texttt{action} are self-explaining. 

A simple example is provided below. It implements the referential integrity \texttt{R.A to S.B} with cascaded deletion.
\begin{lstlisting}
--row-level
Create Trigger Cascade
After Delete On S
Referencing Old Row as O
For Each Row
Delete From R where A = O.B;

--statement-level
Create Trigger Cascade
After Delete On S
Referencing Old Table as OT
Delete From R where A in (select B from OT);
\end{lstlisting}
There are plenty of tricky issues when handling triggers.
\begin{itemize}
\item Row-level v.s. statement-level. \texttt{Old/New Row} can only be referenced when the trigger is row-level. The difference is intuitive when the trigger is activated after the event, but it can get trickier when the activation happens before or instead of the event. 
\item Multiple triggers can be activated at the same time. The order of the activations matters.
\item The action of one trigger may activate another trigger, causing a chaining effect. A trigger that can activate itself leads to self-triggering, and a few triggers can form a cycle. In both cases, termination of the chain becomes an issue. Moreover, a trigger may take several actions that activates different triggers, resulting in nested invocations of triggers. 
\item Conditions can be put either in the \texttt{when} clause or as part of the \texttt{action}. Sometimes the efficiency differs a lot among different choices. 
\end{itemize}
The following complex example using Table \texttt{T(K,V)} (\texttt{K} is the key) demonstrates the subtlety of row-level v.s. statement-level. Each time an insertion is performed on table \texttt{T}, the trigger checks for each newly inserted tuple whether the average of \texttt{V} in table \texttt{T} is smaller than the average of \texttt{V} in the inserted tuples. If so, then \texttt{V} of the current tuple being processed s incremented by 10.
\begin{lstlisting}
Create Trigger IncreaseInserts
After Insert on T
Referencing New Row as NR, New Table as NT
For Each Row
When (select avg(V) from T) < (select avg(V) from NT)
Update T set V=V+10 where K=NR.K;
\end{lstlisting}
There is no statement-level equivalent to this row-level trigger. Moreover, its final state is nondeterministic because its behavior depends on the order in which the inserted tuples are checked. We will illustrate different aspects of trigger with a few examples using SQLite, which differs from the SQL standard in the following aspects:
\begin{itemize}
\item Only row-level trigger is implemented. \texttt{for each row} is implicit if not specified.
\item Triggers are activated immediately after each row-level change instead of after the whole statement as specified in the standard. As a result, there is no reference to \texttt{old table} or \texttt{new table}. Actually there is no referencing clause. \texttt{Old} and \texttt{New} are predefined for \texttt{Old Row} and \texttt{New Row}.
\item The \texttt{action} in a trigger is contained in a \texttt{begin-end} block.
\end{itemize}
\begin{lstlisting}
--All newly inserted students with GPA in (3.3,3.6] apply for geology of Stanford and biology of MIT.
create trigger R1
after insert on Student
for each row
when New.GPA > 3.3 and New.GPA <= 3.6
begin
	insert into Apply values(New.sID,'Stanford','geology',null);
	insert into Apply values(New.sID,'MIT','biology',null);
end;
\end{lstlisting}
\begin{lstlisting}
--cascade deletion in Student to Apply
create trigger R2
after delete on Student
for each row
begin
	delete from Apply where sID = Old.sID;
end;
\end{lstlisting}
\begin{lstlisting}
--cascade update in College to Apply
create trigger R3
after update of cName on College
for each row
begin
	update Apply
	set cName = New.cName where cName = Old.cName;
end;
\end{lstlisting}
\begin{lstlisting}
--enforce key constraint on College
create trigger R4
before insert on College
for each row 
when exists (select * from College where cName = New.cName)
begin
	select raise(ignore);--ignore the current modification
end;

create trigger R5
before update of cName on College
for each row 
when exists (select * from College where cName = New.cName)
begin
	select raise(ignore);--ignore the current modification
end;
\end{lstlisting}
\begin{lstlisting}
--update college name when number of applications > 10
create trigger R6
after insert on Apply
for each row
when (select count(*) from Apply where cName = New.cName) > 10
begin
	update College set cName=cName||'-Done' where cName=New.cName;
end;
\end{lstlisting}
\begin{lstlisting}
--limit sizeHS
create trigger R7
after insert on Student
for each row
when New.sizeHS < 100 or New.sizeHS > 5000
begin
	delete from Student where sID = New.sID;
end;
\end{lstlisting}
Note that \texttt{R7} and \texttt{R1} can be activated at the same time. If we insert a student with GPA = 3.6 and sizeHS = 10000, the applications generated by \texttt{R1} will stay in Apply, while the tuple inserted into Student will be deleted according to \texttt{R7}.\footnote{\texttt{R2} is not taken into account. }
\begin{lstlisting}
--process some applications when enrollment passes threshold of 16000.
create trigger TooMany
after update of enrollment on College
for each row
when (Old.enrollment <= 16000 and New.enrollment > 16000)
begin
	delete from Apply where cName = New.cName and major = 'EE';
	update Apply
		set decision = 'U'
		where cName = New.cName and decision = 'Y';
end;
\end{lstlisting}
\texttt{TooMany} takes action based on the dynamic behavior of the DB, which cannot be done with constraints.

The follow examples use tables \texttt{Ti(A:Int), i = 1,2,3,4}.
\begin{lstlisting}
--self-triggering
creare trigger R1
after insert on T1
for each row
begin
	insert into T1 values(New.A+1);
end;
\end{lstlisting}
The trigger above will end up in infinite loop if no limit is put on the times for which self-triggering can happen. The parameter \texttt{recursive\_triggers} controls this behavior. By default it is off, and the trigger can be activated at most once. If we turn it on with
\begin{lstlisting}
pragma recursive_triggers = on;
\end{lstlisting}
the activation can happen multiple times. A limit has to be put in the \texttt{when} clause to avoid infinite loop error. 
\begin{lstlisting}
--two triggers are actiavetd at the same time and act on the same data
create trigger R1
after insert on T1
for each row
begin
	update T1 set A = 2;
end;

create trigger R2
after insert on T1
for each row
when exists (select * from T1 where A = 2)
begin
	update T1 set A = 3;
end
\end{lstlisting}
The two triggers above are activated at the same time and they both modify the value of \texttt{A}. \texttt{A} may end up with value 2 or 3, depending on the order in which the two triggers are defined.
\ifx\PREAMBLE\undefined
\end{document}
\fi